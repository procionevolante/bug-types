%%%%%%%%%%%%%%%%%%%%%%%%%%%%%%%%%%%%%%%%%
% Dictionary
% LaTeX Template
% Version 1.0 (20/12/14)
%
% This template has been downloaded from:
% http://www.LaTeXTemplates.com
%
% Original author:
% Vel (vel@latextemplates.com) inspired by a template by Marc Lavaud
%
% License:
% CC BY-NC-SA 3.0 (http://creativecommons.org/licenses/by-nc-sa/3.0/)
%
%%%%%%%%%%%%%%%%%%%%%%%%%%%%%%%%%%%%%%%%%

%----------------------------------------------------------------------------------------
%	PACKAGES AND OTHER DOCUMENT CONFIGURATIONS
%----------------------------------------------------------------------------------------

\documentclass[10pt,a4paper,twoside]{article} % 10pt font size, A4 paper and two-sided margins

\usepackage[top=3.5cm,bottom=3.5cm,left=3.7cm,right=4.7cm,columnsep=30pt]{geometry} % Document margins and spacings
\usepackage{pdflscape}

\usepackage[utf8]{inputenc} % Required for inputting international characters
\usepackage[T1]{fontenc} % Output font encoding for international characters

\usepackage{palatino} % Use the Palatino font

\usepackage{microtype} % Improves spacing

\usepackage[bf,sf]{titlesec} % Modify section titles: bold, sans-serif

% @brief Word definition in a dictionary
% @param #1 Word
% @param #2 Sillable division
% @param #3 Word type (noun, verb, ...)
% @param #4 Definition
\newcommand{\entry}[4]{
    \textbf{#1}\ {(#2)}\ \textit{#3}
{
    \begin{quote}
        #4
    \end{quote}
}}

%----------------------------------------------------------------------------------------

\begin{document}

%----------------------------------------------------------------------------------------
%	SECTION A
%----------------------------------------------------------------------------------------

\pagenumbering{gobble} % turn off page numbers

\begin{landscape}

\section*{Bug Types\footnote{Inspired from \texttt{https://en.wikipedia.org/wiki/Heisenbug}}}


    \entry{Bohrbug}{bohr-bug}{Noun}
    {
        A "good, solid bug".
        Like the deterministic Bohr atom model, it does not change its
        behavior and is relatively easily detected.
    }

    \entry{Heisenbug}{hei-sen-bug}{Noun}
    {
        A software bug that seems to disappear or alter its behavior when one
        attempts to study it. \\
        The term is a pun on the name of Werner Heisenberg, the physicist who
        first asserted the observer effect of quantum mechanics, which states
        that the act of observing a system inevitably alters its state.
    }

    \entry{Higgs-bugson}{higgs-bug-son}{Noun}
    {
        \emph{Named after the Higgs boson particle.}
        A bug that is predicted to exist based upon other observed conditions
        (most commonly, vaguely related log entries and anecdotal user reports)
        but is difficult, if not impossible, to artificially reproduce in a
        development or test environment. \\
        $\bullet$ A bug that is obvious in the code (mathematically proven), but which
        cannot be seen in execution (yet difficult or impossible to actually
        find in existence).
    }

    \entry{Hindenbug}{hin-den-bug}{Noun}
    {
        \emph{Named after the Hindenburg disaster.}
        It's a bug with catastrophic behavior.
    }

    \entry{Mandelbug}{man-del-bug}{Noun}
    {
        \emph{Named after Benoît Mandelbrot's fractal.}
        A bug whose causes are so complex it defies repair, or makes its
        behavior appear chaotic or even non-deterministic. \\
        $\bullet$ A bug that exhibits fractal behavior (that is, self-similarity) by
        revealing more bugs (the deeper a developer goes into the code to fix
        it the more bugs they find).
    }

    \entry{Schrödinbug}{schrö-din-bug}{Noun}
    {
        \emph{Named after Erwin Schrödinger and his thought experiment.}
        A bug that manifests itself in running software after a programmer
        notices that the code should never have worked in the first place.
    }

\end{landscape}

\end{document}
